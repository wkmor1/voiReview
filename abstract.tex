
\subsection*{Abstract}\label{abstract}
\addcontentsline{toc}{subsection}{Abstract}

In conservation and natural resource management, scientist and practitioners have begun to realize the importance of valuing information. Information has a central role to play when it comes to making good decisions that will benefit the environment and satisfy management objectives. But there are limits to benefits from more information. Key questions for practitioners are: how much information is warranted for decision making? What kind of information? And what level of certainty is enough before a decision can be made with confidence in the outcome? These questions arise as the information itself comes at some cost. This cost must be weighed against the value of information for the decision at. Decision theoretic tools aimed at information valuing have existed for over half a century but only relatively recently have begun to appear in the conservation and natural resource management literature. Here, we examine a suite of case studies employing value of information (VOI) analyses to applied ecological decision problems. We have surveyed case studies using VOI analysis in the strict sense and compare and contrast them to less formal methods that also, sometimes inadvertently, put a value on ecological data in the context of decision making. Our aim here, is to provide an overview of the use of VOI in the field to date and to glean generalities. We found that the two strands of information valuing, formal and informal, have their own distinct characteristics and the casual reader of either may get a different picture about the value of information if they were only to engage with one or the other. Formal VOI analyses tend to report a low value of information, while informal methods often report larger values. We conjecture that biases stemming from the way that case studies are performed and selected may account for this discrepancy. A feature common to both approaches is that the cost of information is rarely calculated or reported. For greater insight into any generalities on information valuing, future work in conservation sciences should place greater emphasis on information cost and converting costs into the same currency as decision objectives.